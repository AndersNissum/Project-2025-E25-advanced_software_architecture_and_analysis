\documentclass[conference]{IEEEtran}
\IEEEoverridecommandlockouts
% The preceding line is only needed to identify funding in the first footnote. If that is unneeded, please comment it out.
\usepackage{cite}
\usepackage{amsmath,amssymb,amsfonts}
\usepackage{algorithmic}
\usepackage{graphicx}
\usepackage{textcomp}
\usepackage{xcolor}
\usepackage{multirow}
\usepackage{rotating}
\usepackage{mdframed}
\usepackage{hyperref}
\usepackage{tikz}
\usepackage{makecell}
\usepackage{tcolorbox}
\usepackage{amsthm}
%\usepackage[english]{babel}
\usepackage{pifont} % checkmarks
%\theoremstyle{definition}
%\newtheorem{definition}{Definition}[section]

\usepackage{listings}
\lstset
{ 
    basicstyle=\footnotesize,
    numbers=left,
    stepnumber=1,
    xleftmargin=5.0ex,
}

%SCJ
\usepackage{subcaption}
\usepackage{array, multirow}
\usepackage{enumitem}

\def\BibTeX{{\rm B\kern-.05em{\sc i\kern-.025em b}\kern-.08em
    T\kern-.1667em\lower.7ex\hbox{E}\kern-.125emX}}
\begin{document}

%\IEEEpubid{978-1-6654-8356-8/22/\$31.00 ©2022 IEEE}
% @Sune:
% Found this suggestion: https://site.ieee.org/compel2018/ieee-copyright-notice/
% I have added it - you can see if it fulfills the requirements

%\IEEEoverridecommandlockouts
%\IEEEpubid{\makebox[\columnwidth]{978-1-6654-8356-8/22/\$31.00 ©2022 IEEE %\hfill} \hspace{\columnsep}\makebox[\columnwidth]{ }}
                                 %978-1-6654-8356-8/22/$31.00 ©2022 IEEE
% copyright notice added:
%\makeatletter
%\setlength{\footskip}{20pt} 
%\def\ps@IEEEtitlepagestyle{%
%  \def\@oddfoot{\mycopyrightnotice}%
%  \def\@evenfoot{}%
%}
%\def\mycopyrightnotice{%
%  {\footnotesize 978-1-6654-8356-8/22/\$31.00 ©2022 IEEE\hfill}% <--- Change here
%  \gdef\mycopyrightnotice{}% just in case
%}
      
\title{Group Report Template\\
}

\author{
    \IEEEauthorblockN{
        Student 1\IEEEauthorrefmark{1},
        Student 2\IEEEauthorrefmark{1},
        Student 3\IEEEauthorrefmark{1},
        Student 4\IEEEauthorrefmark{1},
        Student 5\IEEEauthorrefmark{1},
    }
    \IEEEauthorblockA{
        University of Southern Denmark, SDU Software Engineering, Odense, Denmark \\
        Email: \IEEEauthorrefmark{1} \textnormal{\{student1,student2,student3,student4,student5\}}@mmmi.sdu.dk
    }
}

%%%%

%\author{\IEEEauthorblockN{1\textsuperscript{st} Blinded for review}
%\IEEEauthorblockA{\textit{Blinded for review} \\
%\textit{Blinded for review}\\
%Blinded for review \\
%Blinded for review}
%\and
%\IEEEauthorblockN{2\textsuperscript{nd} Blinded for review}
%\IEEEauthorblockA{\textit{Blinded for review} \\
%\textit{Blinded for review}\\
%Blinded for review \\
%Blinded for review}
%\and
%\IEEEauthorblockN{3\textsuperscript{nd} Blinded for review}
%\IEEEauthorblockA{\textit{Blinded for review} \\
%\textit{Blinded for review}\\
%Blinded for review \\
%Blinded for review}
%}

%%%%
%\IEEEauthorblockN{2\textsuperscript{nd} Given Name Surname}
%\IEEEauthorblockA{\textit{dept. name of organization (of Aff.)} \\
%\textit{name of organization (of Aff.)}\\
%City, Country \\
%email address or ORCID}

\maketitle
\IEEEpubidadjcol
\begin{abstract}
%%%%%%%%%%%%%%%%%% Max 970 signs without space %%%%%%%%%%%%%%%%%%
% Intro

% Gab
    
% Aim 

% Method

% Results 

\end{abstract}

\begin{IEEEkeywords}
Keyword1, Keyword2, Keyword3, Keyword4, Keyword5
\end{IEEEkeywords}

\section{Introduction and Motivation}
%Introduction and motivate the problem

The structure of the paper is as follows. 
Section \ref{sec:problem} outlines the research question and the research approach. 
%to analyze the research question and evaluate our results.
Section \ref{sec:related_work} describes similar work in the field and how our contribution fits the field.
Section \ref{sec:use_case} presents a production reconfiguration use case.
The use case serves as input to specify a reconfigurability QA requirement in Section \ref{sec:qas}.
Section \ref{sec:middleware_architecture} introduces the proposed reconfigurable middleware software architecture design.
Section \ref{sec:evaluation} evaluates the proposed middleware on realistic equipment in the I4.0 lab and analyzes the results against the stated QA requirement.   

\section{Problem and Approach}

\label{sec:problem}
\emph{Problem.}

\emph{Research questions:}
\begin{enumerate}
    \item  
    \item 
\end{enumerate}

\emph{Approach.}
The following steps are taken to answer this paper's research questions: 
\begin{enumerate}
    \item 
\end{enumerate}


\section{Related work}
\label{sec:related_work}
This Section addresses existing contributions by examining xxx in the I4.0 domain. 
In total, x papers are investigated. 

In \cite{Wan2019Reconfigurable}, experiences are elaborated on a three-layer architecture of a reconfigurable smart factory for drug packing in healthcare I4.0. 

The paper \cite{Yazen2010Ontology} proposes an ontology agent-based architecture for inferring  new configurations to adapt to changes in manufacturing requirements and/or environment.

In \cite{Leitao2016Specification,Angione2017Integration} an architecture for a reconfigurable production system is specified.
Two objectives for reconfiguration and how they can be reached are described.

Several papers \cite{Koren1999Reconfigurable,Koren2010Design,Bortolini2018Reconfigurable} describe reconfigurable manufacturing systems that are cost-effective and responsive to market changes.

All contributions provide valuable knowledge about reconfiguration but lack a study of the software architecture perspective that specifies a quantifiable reconfigurability architectural requirement, a software architecture that adopts the architectural requirements, and evaluates the architectural requirement. 


\section{Use Case}
\label{sec:use_case}
This Section introduces the use cases.

\section{Quality Attribute Scenario}
\label{sec:qas}
This Section introduces the specified x QASes.
The QASes are developed based on the use case.

% Description of the overall architecture designs
% Argue for tactics used to archieve the QASes
% Discuss the trade-offs

\section{Design and Analysis Modelling}
\label{sec:design_and_analysis_modelling}
Design and analysis modelling.

\section{Formal verification and validation}
\label{sec:formal_v_and_v}
Formal verification and validation of system(s).


\section{Evaluation}
\label{sec:evaluation}

% Empirical evaluation
This Section describes the evaluation of the proposed design.
Section \ref{sec:design} introduces the design of the experiment to evaluate the system. 
Section \ref{sec:measurements} identifies the measurements in the system for the experiment.
Section \ref{sec:pilot_test} describes the pilot test used to compute the number of replication in the actual evaluation. 
Section \ref{sec:analysis} presents the analysis of the results from the experiment. 

\subsection{Experiment design}
\label{sec:design}

\subsection{Measurements}
\label{sec:measurements}

\subsection{Pilot test}
\label{sec:pilot_test}

\subsection{Analysis}
\label{sec:analysis}


\section{Conclusion}
\label{sec:conclusion}
Conclusion of the report, discussion and relevant future work.

\subsection{Discussion}
\label{sec:Discussion}

\subsection{Future work}
\label{sec:future_work}

\bibliographystyle{IEEEtran}
\bibliography{references}
\vspace{12pt}
\end{document}
